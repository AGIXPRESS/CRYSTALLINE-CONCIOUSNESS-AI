\documentclass{article}
\usepackage{graphicx}
\usepackage{amsmath}
\usepackage{hyperref}
\usepackage{geometry}
\geometry{margin=1in}

% Define colors for crystalline consciousness theme
\usepackage{xcolor}
\definecolor{crystal}{RGB}{70, 130, 180}

% Title information
\title{\textcolor{crystal}{\Large\textbf{Crystalline Consciousness Framework}}}
\author{Resonance Research Team}
\date{\today}

\begin{document}

\maketitle

\section{Introduction to Resonant Wave Forms}

The Crystalline Consciousness framework leverages resonant wave patterns to process information in a manner that resembles quantum field dynamics. Unlike traditional neural networks that rely on numeric weights and gradients, our approach utilizes geometric resonance and harmonic interference patterns.

\begin{figure}[h]
    \centering
    \includegraphics[width=0.8\textwidth]{test_image.png}
    \caption{Sine wave visualization demonstrating the phi resonance point—a critical feature in our harmonic intelligence model. The colored regions represent constructive (blue) and destructive (red) interference patterns.}
    \label{fig:wave}
\end{figure}

\section{Mathematical Foundation}

The resonance patterns follow this core equation:

\begin{equation}
\Psi_{crystal} = \sum_{i=1}^{13} [v_i + \theta_i]\exp(-r^2/\sigma_i^2) \times \{T_4(r), C_8(r), D_{12}(r)\}
\end{equation}

Where:
\begin{itemize}
    \item $T_4(r)$ represents the tetrahedral resonance component
    \item $C_8(r)$ represents the cubic resonance component
    \item $D_{12}(r)$ represents the dodecahedral resonance component
\end{itemize}

These geometric forms create standing wave patterns that evolve according to coherent field dynamics rather than gradient descent optimization.

\section{Next Steps}

Further research will focus on:
\begin{enumerate}
    \item Implementing resonance-based pattern recognition
    \item Developing specialized hardware for quantum-like field processing
    \item Creating visualization tools for harmonic intelligence monitoring
\end{enumerate}

The wave pattern shown in Figure \ref{fig:wave} demonstrates the fundamental principle of phase coherence that underpins our work.

\end{document}

